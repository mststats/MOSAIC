\documentclass{article}
\usepackage{graphicx}
\usepackage{url}
\usepackage{color}
\usepackage{upquote,textcomp}

\begin{document}
\newcommand{\Fst}{\textrm{F}_{\textrm{st}}}
\newcommand{\hFst}{\hat F_{\textrm{st}}}
\newcommand{\Rst}{\textrm{R}_{\textrm{st}}}
\newcommand{\bm}[1]{\mbox{\boldmath $ #1 $}}

\title{MOSAIC}
\author{Michael Salter-Townshend}

\maketitle

\section{Introduction}
MOSAIC is a tool for modelling multiway admixture using dense genotype data. 
Given a set of potentially admixed haplotypes (targets) and multiple labelled sets of potentially related haplotypes (panels), 
MOSAIC will infer the most recent admixture events occurring in the targets in terms of the panels.

See Salter-Townshend and Myers, 2019 \url{https://doi.org/10.1534/genetics.119.302139} 
for details on the MOSAIC method. 
For results on a large set of human populations see \url{https://maths.ucd.ie/~mst/MOSAIC/HGDP_browser/}. 

It is not necessary that any of the panels are good surrogates for the unseen mixing groups as MOSAIC will infer parameters controlling:
\begin{enumerate}
  \item The stochastic relationship between panels and ancestral groups ($\bm\mu$). 
  \item Timings and ancestry proportions of the admixture events ($\bm\alpha$ and $\bm\lambda$, derived from $\Pi$). 
  \item Recombination rates before and after admixture ($\rho$ and $\bm\Pi$). 
  \item Mutation / error rates for the haplotypes ($\theta$).
\end{enumerate}

Phasing improvements in light of the admixture model are performed and local ancestry along the genome is estimated.

\section{Inputs / Data}
The ``example\_data'' folder packaged with MOSAIC contains example human data for chromosomes 18 to 22.

As inputs MOSAIC requires:
\begin{enumerate}
  \item Phased haplotypes for reference panels and target (admixed) individuals (MOSAIC attempts to detect and correct phasing errors in targets): 
    these should be named ``pop.genofile.chr'' where pop is the population (panel) name and chr is the chromosome index.  
    All entries should be $0,1,?$ indicating ref, alt, or missing entries. The rows are \#snps the columns are and \#haps. Note that there should be no spaces 
    in these files. 
  \item A population names file: ``sample.names'' format unimportant apart from first column should have all the population names.
  \item SNP files: ``snpfile.chr'' with \#snps rows and 6 columns comprising rsID, chr, distance, position, allele ?, allele ?. 
  \item Recombination rates files: ``rates.chr'' 3 rows of \#sites, position, cumulative recombination rate (in centiMorgans). 
\end{enumerate}
Examination of the files in the \texttt{example\_data} folder should make the format of each of the above clear. 

\section{Outputs / Results}
\begin{enumerate}
  \item A folder called \texttt{MOSAIC\_RESULTS} is required to hold log-files (foo.out) and results (foo.RData).  
  \item A folder called \texttt{MOSAIC\_PLOTS} is required to hold the plots created by default by a MOSAIC run.
  \item A folder called \texttt{FREQS} is required to hold the frequencies used to compute $\Fst$ statistics if required.
\end{enumerate}

\section{Simple Simulation Study}

%A real-data example run of mosaic can be done using:\\
%\verb+ Rscript mosaic.R Moroccan example_data/ -a 2 -n 2 -c 18:22+

%or in an interactive R session:\\
%\verb+ mosaic.result=run_mosaic("Moroccan","example_data/",18:22,2,2)+

User defined simulations can be created by specifying a vector of populations:\\
\verb+ Rscript mosaic.R simulated example_data/ -c 18:22 -n 2 -p "English Mandenka"+

or equivalently in an interactive \texttt{R} session:\\
\verb+ mosaic.result=run_mosaic("simulated","example_data/",chrnos=18:22,A=2,NUMI=3,+
\verb+                                pops=c("English","Mandenka"),gens=50) +

Once this has run, MOSAIC will have done:
\begin{enumerate}
  \item Simulated 2-way admixture using English and Mandenkan chromosomes $50$ generations ago.
  \item Read in these simulated chromosomes, along with all available reference panels in the \texttt{example\_data} folder.
  \item Inferred the model parameters $(\bm\mu, \theta, \bm\Pi, \rho)$ via EM.
  \item Corrected phasing errors that scramble local ancestry. 
  \item Estimated 2-way 2-way local ancestry along each chromosome for each admixed individual. 
  \item Estimated $\Fst$ between each ancestral group and each ancestral group and each reference panel. 
  \item Saved key plots to \texttt{MOSAIC\_PLOTS} as PDFs.
  \item Saved key results to \texttt{MOSAIC\_RESULTS}.
\end{enumerate}
The results can be loaded in an \texttt{R} session using:\\
\verb+load("simulated_2way_1-3_18-22_148_60_0.99_100.RData") # model parameters, etc+  \\
\verb+load("localaanc_simulated_2way_1-3_18-22_148_60_0.99_100.RData") # local ancestry estimates+ 

If MOSAIC has been run withing \texttt{R} using the \verb+run_mosaic()+ command above then you can alternatively use\\
\verb+attach(mosaic.result)+ \\
to attach the results for further use. 

\subsection{Plots}
After loading or attaching the results in \texttt{R}, 
each of the plots can be created within \texttt{R} by running:\\
\verb+ plot_all_mosaic(pathout="MOSAIC_PLOTS/",target)+
to output default plots to the folder \texttt{MOSAIC\_PLOTS/}. 
Note that this is already run automatically by default in \verb+run_mosaic()+

Or you can individually run:
\begin{itemize}
  \item Plot the copying matrix:\\
\verb+ ord.Mu<-plot_Mu(Mu,alpha,NL) +
\item Plot co-ancestry curves used to infer event timings:\\
\verb+ plot_coanccurves(acoancs,dr)+
\item Cycle through local ancestry plots (one per target diploid chromosome):\\
\verb+ plot_localanc(chrnos,g.loc,localanc,g.true_anc)+ 
\item Plot the model fit across iterations of thin/phase/EM (if EM is on):\\
\verb+ plot_loglike(extract_log(logfile))+
\end{itemize}

The final plot above plots local ancestry along evenly spaced gridpoints on recombination distances. You can get 
local ancestry estimates at the SNP positions using:\\
\verb+ ans=grid_to_pos(localanc[[1]],loci,g.loc[[1]])+\\
where loci are the SNP positions you'd like to map back to and this is for the first chromosome for which local ancestry has been estimated.


\begin{figure} \centering
    \includegraphics[width=3.0in]{MOSAIC_PLOTS/simulated_2way_6_18-22_148_60_Mu}
    \caption{Simulation Inferred Copying Matrix}\label{simMu}
\end{figure}

\subsection{Other Options}
Note that additional groups will be used as the donor panels but can also be specified manually as follows:\\
\verb+ mosaic.result=run_mosaic("simulated","example_data/",18:22,2,3,+\\
\verb+                          c("English","Mandenka", "French", "Yoruba"))+
%\begin{figure} \centering
%    \includegraphics[width=3.0in]{MOSAIC_PLOTS/simulated_2way_6_18-22_FLAG_60_Mu}
%    \caption{Simulation Inferred Copying Matrix}\label{simMu}
%\end{figure}

%##### OUTPUTS ####################################################################################################
A folder called MOSAIC\_RESULTS is required to hold log-files (foo.out) and results (foo.RData).  
A folder called MOSAIC\_PLOTS is required to hold the plots created by default by a MOSAIC run.
A folder called FREQS is required to hold the frequencies used to compute $\Fst$ statistics if required.
%##################################################################################################################

%########  PARAMETERS INFERRED   ##################################################################################
There are 4 sets of parameters inferred via EM:
	1. PI (prob. of switching between latent ancestries, including switch to same anc; AxA)
	2. rho (prob. of switching haps within each ancestry)
	3. Mu (copying matrix; $Mu[i,j]$ is  prob. of donor from group i given ancestry j; KxA where K is \#donorpops) 
	4. theta (error / mutation; vector length a; prob. of a difference b/w copied and copying haps at a locus)

Note that PI and rho will depend on grid granularity (GpcM); finer grid means lower prob. of switching between gridpoints.
%##################################################################################################################

Finally, use \\
\verb+Rscript mosaic.R --help+\\
on the command line to list all arguments to MOSAIC. 

%FLAG include plots, extraction of localanc, r2 calcs, localanc on SNP positions. 

email \url{michael.salter-townshend@ucd.ie} for help


\begin{thebibliography}{2}

  \bibitem[{Salter-Townshend and Myers}(2019)]{st2019fine}
    \textit{Fine-Scale Inference of Ancestry Segments Without Prior Knowledge of Admixing Groups}
{Salter-Townshend, M {\rm and} Myers, S.R.}
    Genetics: 10.1534/genetics.119.302139.

\end{thebibliography}

\end{document}

\documentclass{article}
\usepackage{graphicx}
\usepackage{color}
\usepackage{upquote,textcomp}

\begin{document}

\title{MOSAIC manual and example run}
\author{Michael Salter-Townshend}

\maketitle

\section{Introduction}
Here is the text of your introduction.

\subsection{Subsection Heading Here}
Write your subsection text here.

\begin{figure}
    \centering
    \includegraphics[width=3.0in]{MOSAIC_PLOTS/simulated_2way_4_18-22_146_60_Mu}
    \caption{Simulation Results}
    \label{simMu}
\end{figure}


The ``example\_data'' folder packaged with MOSAIC contains example human data for chromosomes 18 to 22.

A real-data example run of mosaic can be done using:\\
\verb+ Rscript mosaic.R Moroccan example_data/ -a 2 -n 2 -c 18:22+

or in an interactive R session:\\
\verb+ mosaic.result=run_mosaic("Moroccan","example_data/",18:22,2,2)+

User defined simulations can also be provided by specifying a vector of populations:\\
\verb+ Rscript mosaic.R simulated example_data/ -c 18:22 -n 2 -p "English Mandenka"+

or equivalently in an interactive R session:\\
\verb+ mosaic.result=run_mosaic("simulated","example_data/",18:22,2,2,c("English","Mandenka")) +

Note that additional groups will be used as the donor panels but can also be specified manually as follows:\\
\verb+ mosaic.result=run_mosaic("simulated","example_data/",18:22,2,2,+\\
\verb+                          c("English","Mandenka", "French", "Yoruba"))+

%##### OUTPUTS ####################################################################################################
A folder called MOSAIC\_RESULTS is required to hold log-files (foo.out) and results (foo.RData).  
A folder called MOSAIC\_PLOTS is required to hold the plots created by default by a MOSAIC run.
A folder called FREQS is required to hold the frequencies used to compute Fst statistics if required.
%##################################################################################################################

%########  PARAMETERS INFERRED   ##################################################################################
There are 4 sets of parameters inferred via EM:
	1. PI (prob. of switching between latent ancestries, including switch to same anc; AxA)
	2. rho (prob. of switching haps within each ancestry)
	3. Mu (copying matrix; $Mu[i,j]$ is  prob. of donor from group i given ancestry j; KxA where K is \#donorpops) 
	4. theta (error / mutation; vector length a; prob. of a difference b/w copied and copying haps at a locus)

Note that PI and rho will depend on grid granularity (GpcM); finer grid means lower prob. of switching between gridpoints.
%##################################################################################################################


FLAG include plots, extraction of localanc, r2 calcs, localanc on SNP positions. 

email michael.salter-townshend@ucd.ie for help

\end{document}
